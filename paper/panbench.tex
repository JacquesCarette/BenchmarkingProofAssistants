\documentclass[a4paper,UKenglish,cleveref, autoref, thm-restate,anonymous]{lipics-v2021}
%This is a template for producing LIPIcs articles. 
%for A4 paper format use option "a4paper", for US-letter use option "letterpaper"
%for british hyphenation rules use option "UKenglish", for american hyphenation rules use option "USenglish"
%for section-numbered lemmas etc., use "numberwithinsect"
%for enabling cleveref support, use "cleveref"
%for enabling autoref support, use "autoref"
%for anonymousing the authors (e.g. for double-blind review), add "anonymous"
%for enabling thm-restate support, use "thm-restate"
%for producing a PDF according the PDF/A standard, add "pdfa"

%\pdfoutput=1 %uncomment to ensure pdflatex processing (mandatatory e.g. to submit to arXiv)
%\hideLIPIcs  %uncomment to remove references to LIPIcs series (logo, DOI, ...), e.g. when preparing a pre-final version to be uploaded to arXiv or another public repository

%\graphicspath{{./graphics/}}%helpful if your graphic files are in another directory

\usepackage{todonotes}

%% Macros
\newcommand{\panbench}{Panbench}

\bibliographystyle{plainurl}% the mandatory bibstyle

\title{\panbench} %TODO Please add

%\titlerunning{Dummy short title} %TODO optional, please use if title is longer than one line

\author{Jane {Open Access}}{Dummy University Computing Laboratory, [optional: Address], Country \and My second affiliation, Country \and \url{http://www.myhomepage.edu} }{johnqpublic@dummyuni.org}{https://orcid.org/0000-0002-1825-0097}{(Optional) author-specific funding acknowledgements}%TODO mandatory, please use full name; only 1 author per \author macro; first two parameters are mandatory, other parameters can be empty. Please provide at least the name of the affiliation and the country. The full address is optional. Use additional curly braces to indicate the correct name splitting when the last name consists of multiple name parts.

\author{Joan R. Public\footnote{Optional footnote, e.g. to mark corresponding author}}{Department of Informatics, Dummy College, [optional: Address], Country}{joanrpublic@dummycollege.org}{[orcid]}{[funding]}

\authorrunning{J. Open Access and J.\,R. Public} %TODO mandatory. First: Use abbreviated first/middle names. Second (only in severe cases): Use first author plus 'et al.'

\Copyright{Jane Open Access and Joan R. Public} %TODO mandatory, please use full first names. LIPIcs license is "CC-BY";  http://creativecommons.org/licenses/by/3.0/

\ccsdesc[100]{\textcolor{red}{Replace ccsdesc macro with valid one}} %TODO mandatory: Please choose ACM 2012 classifications from https://dl.acm.org/ccs/ccs_flat.cfm 

\keywords{Dummy keyword} %TODO mandatory; please add comma-separated list of keywords

\category{} %optional, e.g. invited paper

\relatedversion{} %optional, e.g. full version hosted on arXiv, HAL, or other respository/website
%\relatedversiondetails[linktext={opt. text shown instead of the URL}, cite=DBLP:books/mk/GrayR93]{Classification (e.g. Full Version, Extended Version, Previous Version}{URL to related version} %linktext and cite are optional

%\supplement{}%optional, e.g. related research data, source code, ... hosted on a repository like zenodo, figshare, GitHub, ...
%\supplementdetails[linktext={opt. text shown instead of the URL}, cite=DBLP:books/mk/GrayR93, subcategory={Description, Subcategory}, swhid={Software Heritage Identifier}]{General Classification (e.g. Software, Dataset, Model, ...)}{URL to related version} %linktext, cite, and subcategory are optional

%\funding{(Optional) general funding statement \dots}%optional, to capture a funding statement, which applies to all authors. Please enter author specific funding statements as fifth argument of the \author macro.

\acknowledgements{I want to thank \dots}%optional

%\nolinenumbers %uncomment to disable line numbering



%Editor-only macros:: begin (do not touch as author)%%%%%%%%%%%%%%%%%%%%%%%%%%%%%%%%%%
\EventEditors{John Q. Open and Joan R. Access}
\EventNoEds{2}
\EventLongTitle{42nd Conference on Very Important Topics (CVIT 2016)}
\EventShortTitle{CVIT 2016}
\EventAcronym{CVIT}
\EventYear{2016}
\EventDate{December 24--27, 2016}
\EventLocation{Little Whinging, United Kingdom}
\EventLogo{}
\SeriesVolume{42}
\ArticleNo{23}
%%%%%%%%%%%%%%%%%%%%%%%%%%%%%%%%%%%%%%%%%%%%%%%%%%%%%%

\begin{document}

\maketitle

%TODO mandatory: add short abstract of the document
\begin{abstract}
We benchmark four proof assistants (Agda, Idris, Lean and Rocq) through a
single test suite. We focus our benchmarks on the basic features that all
systems based on a similar foundations (dependent type theory) have in common.
We do this by creating an ``over language'' in which to express all the
information we need to be able to output \emph{correct and idiomatic syntax}
for each of our targets. Our benchmarks further focus on ``basic engineering''
of these systems: how do they handle long identifiers, long lines, large
records, large data declarations, and so on.

Our benchmarks reveals both flaws and successes in all systems. We give a
thorough analysis of the results.

We also detail the design of our extensible system. It is designed so that
additional tests and additional system versions can easily be added. While
adding more systems requires more work, the system is also designed to be
extended in this manner.
\end{abstract}

\section{Introduction}
\label{sec:intro}

\todo{this itemized list should be expanded into actual text}
\begin{itemize}
\item benchmark system engineering and scaling
\item benchmarking of anything resembling proofs would be a major research
project
\item the languages (of types/expressions and of declarations) of all 4
are quite similar in their surface expressivity, even though all possess a
myriad of specific features that are unshared
\end{itemize}

\section{Methodology}
\label{sec:method}

\todo[inline]{This is really documenting the 'experiment'. The actual details
of the thinking behind the design is in Section~\ref{sec:design}.}

\todo{this itemized list should be expanded into actual text}
\begin{itemize}
\item single language of tests
\item document the setup of tests, high level
\item document the setup of testing infrastructure, high level
\item linear / exponential scaling up of test 'size'
\end{itemize}

\section{Results}
\label{sec:results}

\section{Discussion}
\label{sec:discuss}

\section{Design of \panbench}
\label{sec:design}

\section{Conclusion}
\label{sec:conc}

%%
%% Bibliography
%%

%% Please use bibtex, 

\bibliography{references}

\end{document}

